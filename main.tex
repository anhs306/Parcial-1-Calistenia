\documentclass{article}
\usepackage[utf8]{inputenc}
\usepackage[spanish]{babel}
\usepackage{listings}
\usepackage{graphicx}
\usepackage{subfig}
\graphicspath{ {images/} }
\usepackage{cite}

\begin{document}

\begin{titlepage}
    \begin{center}
        \vspace*{1cm}
            
        \Huge
        \textbf{Parcial 1 - Calistenia}
            
        \vspace{0.5cm}
        \LARGE
            
        \vspace{1.5cm}
            
        \textbf{Andrés Salazar Hoyos}
            
        \vfill
            
        \vspace{0.8cm}
            
        \Large
        Despartamento de Ingeniería Electrónica y Telecomunicaciones\\
        Universidad de Antioquia\\
        Medellín\\
        Marzo de 2021
            
    \end{center}
\end{titlepage}

\tableofcontents
\newpage
\section{Sección introductoria}\label{intro}
El siguiente informe describirá detalladamente como completar el desafio de las tarjetas, que consiste en llevar 2 tarjetas del punto inicial al punto final (visite la seccion 3 para ver la ilustracion) usando solamente una mano.

\section{Sección de Procedimiento} \label{contenido}
A continuación se describira detalladamente paso por paso como lograr completar el desafio.
\begin{enumerate}
    \item Partiendo del Estado Inicial, con mucho cuidado deslice la hoja hacia un lado y agarre ambas tarjetas con una sola mano.
    \item Con las tarjetas aún en la mano, vuelva a deslizar la hoja hacia la posición en la que estaba inicialmente.
    \item Junte ambas tarjetas, y apoyelas sobre el centro de la hoja de manera vertical y recta, de modo que la parte mas corta de la tarjeta quede apoyada sobre la hoja, y los laterales de la tarjeta mirando a su cuerpo.
    \item Coloque la yema del dedo pulgar en el final del lateral de una de las dos tarjetas (si está usando la mano derecha, entonces la tarjeta de la derecha, y vicerversa), la yema del dedo indice en el centro de la parte superior de las dos tarjetas, y la yema del dedo anular en el final del lateral de la msima tarjeta que esta sosteniendo el dedo pulgar.
    \item  Lentamente sin mover el dedo indice, y de manera delicada, abrir un poco la tarjeta que está siendo sostenida por el dedo pulgar y anular, usando estos mismos dedos.
    \item Si siente que ya están equilibradas, suelte la mano, si no, puede soltar los dedos pulgar y anular, y moverlas un poco desde arriba con el dedo indice para lograrlo. Por otra parte, si las tarjetas caen, regrese al paso 3 e intentelo de nuevo.
\end{enumerate}

\section{Inclusión de imágenes} \label{imagenes}


\end{document}
